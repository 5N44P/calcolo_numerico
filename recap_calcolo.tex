\documentclass{article}
\usepackage[utf8]{inputenc}
\usepackage[italian]{babel}
\usepackage{geometry}
\usepackage{amsfonts} 
\usepackage{ccicons}
\usepackage{url}
\usepackage{hyperref}

\geometry{left = 2 cm, right = 2 cm, bottom = 2 cm, top = 2 cm}
\setlength{\parindent}{0em}
\hypersetup{
	colorlinks=true,
	urlcolor=blue
}

\title{Calcolo Numerico}
\author{Valerio Nappi - me@valerionappi.it}
\date{AA 2019/2020}

\begin{document}
\maketitle

\section{Riguardo al documento}
Quest'opera è distribuita con Licenza Creative Commons - Attribuzione Non commerciale 4.0 Internazionale \ccbynceu  \newline
Link repository di GitHub: \url{https://github.com/5N44P/calcolo_numerico} link diretto \href{https://github.com/5N44P/calcolo_numerico/blob/master/recap_calcolo.pdf}{qua}. \newline 

\section{Rappresentazione numerica nei calcolatori}
\begin{itemize}
	\item TODO
\end{itemize}

\section{Ricerca di zeri}

\subsection{Metodo di bisezione}

\subsection{Metodi di punto fisso}

\subsection{Metodo di Newton}


\section{Sistemi lineari}

\subsection{Metodo di Cramer}

\subsection{Metodo di Gauss e pivoting}

\subsection{Decomposizione LU}

\subsection{Decomposizione simmetrica}

\subsection{Decomposizione di Cholesky}



\end{document}
